\documentclass[12pt]{ctexart}

% 基本排版与数学支持
\usepackage[a4paper,margin=2.2cm]{geometry}
\usepackage{amsmath,amssymb}
\usepackage{graphicx}
\usepackage{xcolor}
\usepackage{listings}
\usepackage{tcolorbox}
\tcbuselibrary{skins,breakable}

% 配色:亮色、清新天蓝风格
% 背景偏白,主体文字深灰,强调色为天蓝
\definecolor{pagebg}{HTML}{F9FAFF}   % 页面背景(极浅蓝白)
\definecolor{cardbg}{HTML}{E0F2FE}   % 说明框背景(淡天蓝)
\definecolor{codebg}{HTML}{F1F5F9}   % 代码背景(浅灰蓝)
\definecolor{accent}{HTML}{0EA5E9}   % 主强调色:天蓝
\definecolor{accent2}{HTML}{38BDF8}  % 辅强调色:更亮的天蓝
\definecolor{textgray}{HTML}{0F172A} % 主体文字:深蓝灰

\pagecolor{pagebg}
\color{textgray}

% C++ 代码风格
\lstdefinestyle{cppstyle}{%
  language=C++,
  backgroundcolor=\color{codebg},
  basicstyle=\ttfamily\small\color{textgray},
  keywordstyle=\color{accent}\bfseries,
  commentstyle=\color{gray!70!black},
  stringstyle=\color{accent2},
  numberstyle=\tiny\color{gray},
  numbers=left,
  stepnumber=1,
  numbersep=8pt,
  tabsize=4,
  showstringspaces=false,
  breaklines=true,
  columns=fullflexible
}

% 算法说明与代码块外框
\newtcolorbox{algobox}[1][]{%
  breakable,
  enhanced,
  colback=cardbg,
  colframe=accent,
  boxrule=0.9pt,
  arc=3pt,
  left=8pt,right=8pt,top=6pt,bottom=6pt,
  % 标题区域只设置字体即可,避免误用 tikz 的 colframe 关键字
  title style={font=\bfseries\small},
  coltitle=accent,
  #1
}

\newtcolorbox{codebox}[1][]{%
  breakable,
  enhanced,
  colback=codebg,
  colframe=accent!70,
  boxrule=0.9pt,
  arc=3pt,
  left=4pt,right=4pt,top=4pt,bottom=4pt,
  #1
}

\begin{document}

\begin{center}
  {\Huge 盛最多水的容器 \par}
  \vspace{0.4em}
  {\large 三种时间复杂度的对比题解}\par
  \vspace{0.6em}
  {\small Problem ID: 3123 / ZJUT}
\end{center}

\vspace{0.8em}

\begin{algobox}[title={题目简述}]
给定数组 \(\text{height}[0\ldots n-1]\),第 \(i\) 条竖线的高度为 \(\text{height}[i]\)。
在所有两条竖线组成的容器中,求最大面积:
\[
  \text{area}(i,j) = (j-i)\cdot\min(\text{height}[i],\text{height}[j]).
\]
本题有三个典型解法:暴力 \(O(n^2)\)、排序贪心 \(O(n\log n)\)、双指针 \(O(n)\)。
\end{algobox}

\section*{一、暴力枚举 \(O(n^2)\)}

\begin{algobox}[title={算法思路}]
\begin{itemize}
  \item 用两重循环枚举所有下标对 \((i,j)\),其中 \(0\le i<j<n\)。
  \item 对每一对 \((i,j)\) 计算面积 \((j-i)\cdot\min(h_i,h_j)\),维护最大值。\\
  这里 \(h_i=\text{height}[i]\)。
  \item 时间复杂度为 \(O(n^2)\),当 \(n\) 稍大时代码会偏慢,但逻辑最直观,适合作为入门思路。
\end{itemize}
\end{algobox}

\begin{codebox}
\lstset{style=cppstyle}
\begin{lstlisting}
#include <bits/stdc++.h>
using namespace std;

int n, a[105], ans;

int main() {
    cin >> n;
    for (int i = 0; i < n; i++) cin >> a[i];

    // 枚举每一种 (i, j) 组合
    for (int i = 0; i < n; i++) {
        for (int j = i + 1; j < n; j++) {
            ans = max(ans, (j - i) * min(a[i], a[j]));
        }
    }

    cout << ans;
    return 0;
}
\end{lstlisting}
\end{codebox}

\section*{二、排序贪心 \(O(n\log n)\)}

\begin{algobox}[title={算法思路}]
从另一个角度看问题:
\begin{itemize}
  \item 将所有下标按高度从大到小排序,依次处理第 \(k\) 高的竖线。处理到它时,之前出现的竖线高度都不低于它。\\
  因此,这根竖线一定是当前容器的“短板”。
  \item 容器高度固定为当前高度,只需尽量拉大宽度。\\
  维护目前为止出现过的最小下标 \(mn\) 与最大下标 \(mx\),
  对当前下标 \(idx\) 计算
  \[
    \text{area} = h_{idx} \cdot \max\bigl(|idx-mn|, |idx-mx|\bigr).
  \]
  \item 每个下标仅参与一次排序和一次更新,整体复杂度为 \(O(n\log n)\)。
\end{itemize}
\end{algobox}

\begin{codebox}
\lstset{style=cppstyle}
\begin{lstlisting}
#include <bits/stdc++.h>
using namespace std;

int n, a[105], p[105], ans;

int main() {
    cin >> n;
    for (int i = 0; i < n; i++) {
        cin >> a[i];
        p[i] = i;           // 记录原始下标
    }

    // 按高度从大到小排序下标
    sort(p, p + n, [&](int x, int y) {
        return a[x] > a[y];
    });

    int mn = (int)1e9, mx = -1;
    for (int i = 0; i < n; i++) {
        int idx = p[i];     // 当前柱子的原始位置
        if (mx != -1) {
            int width = max(abs(idx - mn), abs(idx - mx));
            ans = max(ans, a[idx] * width);
        }
        mn = min(mn, idx);
        mx = max(mx, idx);
    }

    cout << ans;
    return 0;
}
\end{lstlisting}
\end{codebox}

\section*{三、双指针法(最优解) \(O(n)\)}

\begin{algobox}[title={算法核心与证明直观}]
设左右指针初始为 \(L=0, R=n-1\),当前面积为
\[
  S = (R-L)\cdot\min(h_L,h_R).
\]
\begin{itemize}
  \item 若 \(h_L < h_R\),则高度由左边决定。如果此时只移动右指针向左,宽度变小,高度最多仍为 \(h_L\),面积不会变大,所以这些状态可以全部剪枝;因此我们只需要右移短的一侧:\(L\leftarrow L+1\)。
  \item 同理,当 \(h_L \ge h_R\) 时,左侧再怎么不动,右侧往左移动都不可能得到比当前更大的面积,应当移动右指针:\(R\leftarrow R-1\)。
  \item 每一步都缩小区间,指针一共移动 \(O(n)\) 次,即可穷尽所有“有希望”的状态。
\end{itemize}
\end{algobox}

\begin{codebox}
\lstset{style=cppstyle}
\begin{lstlisting}
#include <bits/stdc++.h>
using namespace std;

int n, a[105], l, r, mx;

int main() {
    cin >> n;
    for (int i = 0; i < n; i++) cin >> a[i];

    l = 0;
    r = n - 1;

    // 只要没有相遇,就持续计算并尝试移动短板
    while (l < r) {
        mx = max(mx, (r - l) * min(a[l], a[r])); // 当前面积

        // 谁更短就移动谁,寻找更高的一侧
        if (a[l] < a[r]) {
            l++;
        } else {
            r--;
        }
    }

    cout << mx;
    return 0;
}
\end{lstlisting}
\end{codebox}

\section*{四、通过测试截图}

\begin{algobox}[title={说明}]
上述三份 C++ 代码均在评测系统上编译运行通过,状态为 \texttt{Accepted}。下图给出其中双指针解法的提交截图作为证明。
\end{algobox}

\begin{center}
  \includegraphics[width=0.7\textwidth]{accepted.png}\\[0.4em]
  {\small 图:双指针解法提交记录,状态为 Accepted}
\end{center}

\end{document}