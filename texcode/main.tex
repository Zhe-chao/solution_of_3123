\documentclass[11pt, a4paper]{ctexart}

% =============================================================================
% 宏包引入
% =============================================================================
\usepackage[top=2.5cm, bottom=2.5cm, left=2.5cm, right=2.5cm]{geometry} % 标准页边距
\usepackage{amsmath, amssymb, amsfonts}   % 数学公式
\usepackage{graphicx}                     % 图片
\usepackage{xcolor}                       % 颜色
\usepackage{listings}                     % 代码
\usepackage{booktabs}                     % 三线表
\usepackage{fancyhdr}                     % 页眉页脚
\usepackage[colorlinks=true, linkcolor=NavyBlue, urlcolor=NavyBlue]{hyperref} 
\usepackage{tcolorbox}                    % 文本框核心
\tcbuselibrary{skins, breakable, listings} % tcolorbox 扩展库

% =============================================================================
% 配色方案 (Scientific Deep Blue Theme)
% =============================================================================
\definecolor{NavyBlue}{HTML}{003f5c}      % 主色:深海军蓝
\definecolor{AccentBlue}{HTML}{2f4b7c}    % 辅色
\definecolor{BgGray}{HTML}{f5f7fa}        % 页面背景灰
\definecolor{CodeBg}{HTML}{282c34}        % 代码背景:深色IDE风

% --- 代码高亮配色 (优化版) ---
\definecolor{StringColor}{HTML}{98c379}   % 字符串:清新的绿色
\definecolor{KeywordColor}{HTML}{c678dd}  % 关键字:紫色
\definecolor{NumberColor}{HTML}{d19a66}   % 数字:橙色
% [修改点 1] 注释颜色改为高亮浅灰,确保深色背景下清晰可见
\definecolor{CommentColor}{HTML}{abb2bf}  

% =============================================================================
% 样式设置
% =============================================================================

% 1. 全局段落设置
\setlength{\parindent}{0pt}       % 取消首行缩进
\setlength{\parskip}{0.6em}       % 增加段间距

% 2. 页眉页脚
\pagestyle{fancy}
\fancyhf{}
\lhead{\small \bfseries Problem 3123: Container With Most Water}
\rhead{\small \thepage}
\renewcommand{\headrulewidth}{0.5pt}

% 3. 代码块样式 (C++)
\lstdefinestyle{mycpp}{
    language=C++,
    basicstyle=\footnotesize\ttfamily\color{white}, % 默认文字白色
    backgroundcolor=\color{CodeBg},
    keywordstyle=\color{KeywordColor}\bfseries,
    commentstyle=\color{CommentColor}\itshape, % 使用新的高亮注释色
    stringstyle=\color{StringColor},
    numberstyle=\tiny\color{CommentColor},
    identifierstyle=\color{white},
    numbers=left,
    stepnumber=1,
    numbersep=10pt,
    tabsize=4,
    showspaces=false,
    showstringspaces=false,
    breaklines=true,
    frame=none,
    captionpos=b
}

% 4. 题目描述框
\newtcolorbox{problembox}[1][]{
    enhanced,
    colback=BgGray,          
    colbacktitle=NavyBlue,   
    coltitle=white,          
    colframe=NavyBlue,       
    fonttitle=\bfseries,
    title={#1},
    arc=3pt,
    boxrule=0.5mm,
    left=10pt, right=10pt, top=8pt, bottom=8pt,
    shadow={1mm}{-1mm}{0mm}{black!10}
}

% 5. 算法模块框
\newtcolorbox{algobox}[1][]{
    enhanced,
    breakable,
    colback=white,
    colframe=AccentBlue,
    leftrule=3pt,            
    rightrule=0pt, toprule=0pt, bottomrule=0pt,
    arc=0pt,
    coltitle=AccentBlue,
    fonttitle=\bfseries\large,
    title={#1},
    parbox=false,            
    detach title,            
    before upper={\tcbtitle\par\vspace{0.3em}} 
}

% 6. 代码容器
\newtcblisting{codebox}{
    enhanced,
    colback=CodeBg,
    colframe=CodeBg,
    arc=5pt,
    listing only,
    listing style=mycpp,
    left=5pt, right=5pt, top=5pt, bottom=5pt
}

% =============================================================================
% 正文开始
% =============================================================================
\begin{document}

% --- 标题区域 ---
\begin{center}
    {\huge \bfseries \color{NavyBlue} 盛最多水的容器} \\[0.5em]
    {\Large \textit{Container With Most Water}} \\[1em]
    {\color{gray} Problem ID: 3123 \quad | \quad Source: ZJUT OJ}
    \vspace{1em}
    \hrule height 1pt
\end{center}

% --- 题目描述 ---
\begin{problembox}[题目描述]
    给定一个长度为 $n$ 的整数数组 $\texttt{height}$。有 $n$ 条垂线,第 $i$ 条线的两个端点是 $(i, 0)$ 和 $(i, \text{height}[i])$。
    
    找出其中的两条线,使得它们与 $x$ 轴共同构成的容器可以容纳最多的水。
    
    \textbf{目标}:返回容器可以储存的最大水量。
    \begin{equation*}
        \text{Area} = (j - i) \times \min(\text{height}[i], \text{height}[j])
    \end{equation*}
\end{problembox}


\vspace{1em}

% --- 方法一 ---
\section*{一、暴力枚举法 (Brute Force)}

\begin{algobox}[算法思路]
    直接使用两层循环枚举所有可能的左边界 $i$ 和右边界 $j$。对于每一对 $(i, j)$,计算其面积并更新最大值。虽然逻辑简单,但计算量极大。
\end{algobox}

\begin{codebox}
// 核心逻辑片段
for (int i = 0; i < n; i++) {
    for (int j = i + 1; j < n; j++) {
        // 宽度 * 高度瓶颈
        ans = max(ans, (j - i) * min(a[i], a[j]));
    }
}
\end{codebox}

% --- 方法二 ---
\section*{二、排序贪心法 (Sorting Strategy)}

\begin{algobox}[算法思路]
    将柱子按\textbf{高度从大到小}排序。处理第 $k$ 高的柱子时,它一定是当前构成的容器的“短板”。为了使面积最大,我们需要找到原数组中距离当前柱子\textbf{最远}的且高度大于等于它的柱子。通过维护已遍历柱子的最小和最大索引即可实现。
\end{algobox}

\begin{codebox}
// 记录原始下标并按高度降序排序
sort(p, p + n, [&](int x, int y) { return a[x] > a[y]; });

int mn = 1e9, mx = -1;
for (int i = 0; i < n; i++) {
    int idx = p[i]; // 获取原始位置
    if (mx != -1) {
        // 计算到最左端和最右端的最大跨度,因为这些柱子都比当前高
        int width = max(abs(idx - mn), abs(idx - mx)); 
        ans = max(ans, a[idx] * width);
    }
    mn = min(mn, idx); // 更新已见过的最左边界
    mx = max(mx, idx); // 更新已见过的最右边界
}
\end{codebox}

% --- 方法三 ---
\section*{三、双指针法 (Two Pointers - Optimal)}

\begin{algobox}[算法核心与证明]
    这是本题的最优解。定义双指针 $L=0, R=n-1$。
    每次移动两个指针中\textbf{较短}的那一个。
    
    \textbf{证明}:假设 $h[L] < h[R]$,当前面积由 $h[L]$ 决定。如果保持 $L$ 不动而左移 $R$,容器宽度变小,而高度受限于 $h[L]$ 绝不可能增加,因此面积必然减小。为了寻找更大的面积,必须舍弃当前的短板 $L$,尝试 $L+1$。
\end{algobox}

\begin{table}[h]
\centering
\caption{算法性能对比}
\begin{tabular}{llcl} 
\toprule
\textbf{方法} & \textbf{时间复杂度} & \textbf{空间复杂度} & \textbf{评价} \\ 
\midrule
暴力枚举 & $O(n^2)$ & $O(1)$ & 数据量大时超时 (TLE) \\
排序贪心 & $O(n \log n)$ & $O(n)$ & 较优,但在原本有序时非最优 \\
\textbf{双指针} & \textbf{$O(n)$} & \textbf{$O(1)$} & \textbf{最优解,一次遍历} \\
\bottomrule
\end{tabular}
\end{table}

\begin{codebox}
#include <bits/stdc++.h>
using namespace std;

int main() {
    int n;
    cin >> n;
    vector<int> a(n);
    for(int i=0; i<n; i++) cin >> a[i];

    int l = 0, r = n - 1, mx = 0;
    
    while (l < r) {
        // 计算当前面积
        int h = min(a[l], a[r]);
        mx = max(mx, (r - l) * h);
        
        // 贪心策略:谁短移动谁
        if (a[l] < a[r]) l++;
        else r--;
    }
    
    cout << mx << endl;
    return 0;
}
\end{codebox}

\newpage

% --- 结果验证与扩展阅读 ---
% [修改点 2] 增加了评测截图和博客引流部分
\section*{四、结果验证与扩展阅读}

\begin{algobox}[评测结果]
    上述三种方法中,双指针法与排序贪心法均可通过测试。以下为最终提交通过的截图:
    
    \begin{center}
        % 请确保文件夹中有 accepted.png
        \includegraphics[width=0.85\textwidth]{accepted.png} 
    \end{center}
\end{algobox}

\vspace{1em}

\begin{algobox}[算法可视化演示]
    为了更直观地理解“双指针”是如何移动并剪枝搜索空间的,我在博客中制作了详细的\textbf{动态演示 (Animation)}。
    
    \begin{center}
        % 请确保文件夹中有 blog.png
        \includegraphics[width=0.85\textwidth]{blog.png}
        
        \vspace{1em}
        
        % 显眼的跳转链接
        \href{https://zhe-chao.github.io/solution_of_3123/solution.html}{
            \fbox{\bfseries \Large $\rightarrow$ 点击此处跳转博客查看动画演示 $\leftarrow$}
        }
    \end{center}
\end{algobox}

\vspace{2em}
\hrule
\vspace{1em}
{\small \hfill \color{gray}  $\cdot$ \quad \today}

\end{document}